\documentclass[11pt,letterpaper]{article}
\usepackage[margin=1in]{geometry}
\usepackage{parskip}
\usepackage{hyperref}
\usepackage{listings}
\usepackage{xcolor}
\usepackage{booktabs}

\lstset{
  basicstyle=\small\ttfamily,
  keywordstyle=\color{blue!70!black},
  commentstyle=\color{gray},
  stringstyle=\color{red!60!black},
  breaklines=true,
  frame=single,
  xleftmargin=1em,
  framexleftmargin=0.5em,
}

\title{hadlink: A Minimal, High-Assurance URL Redirection Service}
\author{}  % TODO: Add author name
\date{}

\begin{document}
\maketitle

\begin{abstract}
% TODO: Write abstract (~100 words)
% Core argument: URL shorteners are deceptively small systems with disproportionate
% blast radius. Most implementations conflate policy and mechanism, trust input too
% broadly, and are difficult to reason about. hadlink is a URL redirection service
% designed for infrastructure use cases (CI/CD, QR codes, SMS) where deterministic
% behavior, security, and auditability matter more than feature breadth. Its core
% validation and encoding logic is formally verified using SPARK Ada, while the
% service layer is composed in Haskell. This paper describes the system's design,
% assurance model, and deployment approach.
\end{abstract}

% ============================================================================
\section{Problem Statement}
% ============================================================================

% "This is a deceptively small problem with disproportionate blast radius."
%
% URL shorteners are frequently:
% - Abuse targets (spam, phishing, malware distribution)
% - Redirectors of last resort (anything that accepts a URL becomes an open relay)
% - Deployed casually (minimal validation, no formal reasoning about behavior)
%
% Most implementations:
% - Conflate policy and mechanism (validation mixed with routing mixed with storage)
% - Trust input too much (accepting arbitrary URLs, including private addresses)
% - Are hard to reason about (no formal specification, behavior varies by input)
%
% The problem is small enough to be tractable for formal methods, but consequential
% enough that the effort is justified.
%
% TODO: Write 2-3 paragraphs.

% ============================================================================
\section{Design Goals and Non-Goals}
% ============================================================================

% This section signals scope maturity.
%
% \subsection{Goals}
% - Deterministic behavior (same URL always produces same short code)
% - Minimal trusted computing base
% - Explicit failure modes (enumerated error codes, no silent failures)
% - Deployable on constrained systems (single binary, SQLite, no external services)
%
% \subsection{Non-Goals}
% - Marketing analytics or click tracking
% - User accounts or dashboards
% - Custom aliases
% - Dynamic policy engines
% - JavaScript redirects or link previews
% - Feature parity with SaaS shorteners (Bitly, TinyURL)
%
% "These are conscious trade-offs, not omissions."
%
% Non-goals are documented in the repository and enforced in the contribution
% policy. Pull requests that conflict with non-goals are closed with explanation.
%
% TODO: Write 2-3 paragraphs.

% ============================================================================
\section{Architecture}
% ============================================================================

% Two-layer design with intentional separation:
%
% \subsection{SPARK Core}
% Formally verified logic for:
% - URL canonicalization (scheme validation, private address rejection,
%   credentials detection)
% - Short code generation (HMAC-SHA256 via SPARKNaCl, Base62 encoding)
%
% Contains no IO, networking, storage, concurrency, or configuration parsing.
% 137 proof obligations, all verified automatically by CVC5.
%
% \subsection{Haskell Service Layer}
% Composes the system: HTTP (Warp), storage (SQLite), rate limiting, structured
% logging, proof-of-work validation. Property-tested via Hedgehog.
%
% \subsection{FFI Boundary}
% Minimal and frozen (API version 1). Three exported C functions:
% hadlink\_canonicalize, hadlink\_make\_short\_code, hadlink\_api\_version.
% FFI layer is explicitly outside SPARK verification (pragma SPARK\_Mode Off).
% A freeze test ensures the interface does not change without a version bump.
% - The FFI is thin by design
% - HTTP (Warp) is selected because it is thin
%
% \subsection{Split Binaries}
% Two separate executables:
% - \texttt{hadlink-redirect}: Read-only, fast, no SPARK dependency. Exposed
%   to network.
% - \texttt{hadlink-shorten}: Write path, includes SPARK FFI, rate limiting,
%   PoW. Restricted to LAN/VPN.
%
% The redirect binary does not link libHadlink\_Core.so. This is least-privilege
% at the binary level.
%
% \subsection{Storage}
% SQLite in WAL mode. Append-only on the write path. Read-only access for
% redirect. Schema: short\_code (PK) -> canonical\_url, created\_at.
% INSERT OR IGNORE provides idempotent creation (deterministic codes mean
% duplicate requests are harmless).
%
% TODO: Write section prose, optionally include architecture diagram.

% ============================================================================
\section{Formal Assurance}
% ============================================================================

% What is proven:
% - All stored URLs have valid HTTP/HTTPS scheme
% - No stored URL contains credentials (user:pass@host)
% - No stored URL points to a private address (RFC 1918, RFC 4193, link-local)
% - Short codes are exactly 8 Base62 characters
% - All array accesses are in bounds
% - No integer overflow in encoding logic
%
% What is NOT proven:
% - FFI marshaling correctness (pragma SPARK_Mode Off)
% - Haskell service behavior (property-tested, not formally verified)
% - SQLite behavior (trusted dependency)
% - Network-level properties (TLS, DNS)
%
% Assumptions made explicit:
% - 2 pragma Assume statements, both in a ghost lemma, both documenting
%   pure function determinism (A = B implies f(A) = f(B))
% - Zero assumes in business logic
%
% \begin{table}[h]
% \centering
% \begin{tabular}{lr}
% \toprule
% Check type & Count \\
% \midrule
% Range checks & 31 \\
% Overflow checks & 30 \\
% Precondition & 29 \\
% Index checks & 11 \\
% Loop invariants & 18 \\
% Postcondition & 9 \\
% Other & 9 \\
% \midrule
% Total & 137 \\
% \bottomrule
% \end{tabular}
% \end{table}
%
% The project references DO-278A (SIL-3) integrity objectives as an architectural
% guide, but is not certified. This is a single-developer project without
% independent verification resources.
%
% TODO: Write 2-3 paragraphs with table.

% ============================================================================
\section{Deployment}
% ============================================================================

% This section is rare in whitepapers and valuable.
%
% Three deployment methods:
% - Docker (pre-built binaries via GitHub Releases, or build from source)
% - Systemd (direct installation with security-hardened unit files)
% - Arch Linux AUR (hadlink-bin package)
%
% Systemd hardening: NoNewPrivileges, ProtectSystem=strict, ProtectHome,
% PrivateTmp, MemoryDenyWriteExecute, resource limits (128M shorten, 64M redirect).
%
% Configuration minimalism: environment variables only, no config file parser,
% no YAML/TOML. Secret management via file-based EnvironmentFile.
%
% Failure behavior: both services restart on failure (RestartSec=5s). Redirect
% service is stateless and recovers instantly. Shorten service holds no
% in-memory state beyond rate limiter (STM-based, rebuilt on restart).
%
% Pre-built binaries available from GitHub Releases---no build toolchain
% required for deployment.
%
% TODO: Write 2-3 paragraphs.

% ============================================================================
\section{Limitations and Future Work}
% ============================================================================

% This section earns credibility.
%
% Intentional limitations:
% - SQLite scaling envelope (sufficient for infrastructure use, not for
%   high-volume marketing workloads)
% - No built-in abuse detection (rate limiting and PoW mitigate, but do not
%   detect malicious URLs)
% - Single-node only (no replication, no distributed deployment)
% - No URL expiration or deletion (append-only by design)
%
% Potential future work:
% - LMDB backend for higher read throughput
% - TLS termination (currently relies on reverse proxy)
% - Metrics export (Prometheus endpoint)
%
% "These are conscious trade-offs, not omissions. Each limitation preserves
% a property the system depends on."
%
% TODO: Write 1-2 paragraphs.

% ============================================================================
\section{Conclusion}
% ============================================================================

% Reinforce:
% - Correctness over features
% - Simplicity as a security primitive
% - Formal methods are tractable for small infrastructure services
% - The approach---SPARK core, thin FFI, Haskell composition---is applicable
%   to other services with similar trust requirements
%
% TODO: Write 1-2 paragraphs.

\end{document}

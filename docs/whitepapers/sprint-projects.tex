\documentclass[11pt,letterpaper]{article}
\usepackage[margin=1in]{geometry}
\usepackage{parskip}
\usepackage{hyperref}
\usepackage{enumitem}

\title{From Coursework to Craft:\\Single-Sprint Projects as Engineering Practice}
\author{}  % TODO: Add author name
\date{}

\begin{document}
\maketitle

\begin{abstract}
% TODO: Write abstract (~100 words)
% Core argument: Undergraduate engineering coursework builds breadth and teaches
% students to finish. Short-timeline personal projects---constrained to one or two
% weeks---complement this by training scope discipline, invariant-driven design,
% and the judgment to stop. This paper compares both modes of learning and argues
% that single-sprint projects are effective deliberate practice for professional
% engineering skills that coursework alone does not emphasize.

% note: coursework alone is insufficient, this is a major case for frequent project
% work in academia prior to jumping into professional work.
\end{abstract}

% ============================================================================
\section{Introduction}
% ============================================================================

% Thesis: Fast project turnover in undergraduate engineering builds breadth,
% and single-sprint personal projects build judgment and carry momentum.
%
% Frame this as complementary, not adversarial. College teaches you to finish
% under constraints set by others. Sprint projects teach you to set your own
% constraints and live with them. This cultivates senior engineers early.
%
% Key observation: "You learn how to finish. You rarely learn how to stop."
%
% TODO: Write 2-3 paragraphs introducing the comparison.

% ============================================================================
\section{What Coursework Does Well}
% ============================================================================

% Be fair and specific. ECE programs in particular expose students to:
%
% - Multiple parallel projects across domains (digital, analog, embedded, software)
% - Fixed deadlines with moving scopes
% - Collaboration and communication under pressure
% - Foundational knowledge across a broad surface area
%
% These are genuinely valuable. The point is not that coursework is insufficient---
% it is that it trains a specific set of muscles. Programs that encourage
% project work alongside coursework enhance development of real-world skills:
%
% - Time & scope management
% - Design by invariants (project rubric requirements)
% - Documentation of friction points and future work
%
% Acknowledge: grades incentivize completion, which is a real and useful skill.
% Completing a project under someone else's constraints is not trivial.
%
% TODO: Write 2-3 paragraphs.

% ============================================================================
\section{What Coursework Does Not Emphasize}
% ============================================================================

% Three skills that sprint projects train more directly:
%
% \subsection{Scope Discipline}
%
% In coursework, scope creep is expected---requirements change, features get added
% to meet rubric items. Cutting features feels like failure.
%
% In practice, cutting features is the job. Most engineering value comes from what
% you don't build. A constrained personal project forces this: there is no rubric
% to optimize for, no TA to ask for an extension. You must decide what matters and
% what doesn't, then commit.
%
% \subsection{Invariant-Driven Design}
%
% Course projects emphasize outputs: demonstrations, reports, "it works on the
% happy path." Sprint projects reward defining invariants early---what must always
% be true---and designing around failure modes. This is where professional
% engineering actually lives.
%
% Introduce the concept of "design by non-goals": explicitly documenting what the
% system will never do, and enforcing those boundaries. This is uncommon in
% coursework but standard in mature engineering organizations.
%
% \subsection{Taste and Restraint}
%
% You cannot brute-force a one-week project with complexity. You must choose
% conservative tools, reject clever abstractions, and bias toward debuggability.
% This is professional taste forming---and it is difficult to practice in a
% semester-long project where complexity can always be deferred.
%
% TODO: Write 3-4 paragraphs across subsections.

% ============================================================================
\section{Single-Sprint Projects as Deliberate Practice}
% ============================================================================

% Define the format: 1--2 weeks, self-imposed scope, production-quality target.
%
% Characteristics:
% - No grade safety net---every decision reflects directly on you
% - Design trade-offs are permanent (no "fix it next sprint")
% - Forces early architectural clarity
% - Simulates real sprint timelines
%
% Compare to coursework sprints: in college, a "sprint" often means a burst
% of effort before a deadline. In industry, a sprint is a bounded commitment
% with explicit deliverables and explicit cuts. Single-sprint personal projects
% practice the latter.
%
% This is also where time management and expectation management are learned
% organically. There is no project manager---you are the project manager.
%
% TODO: Write 2-3 paragraphs.

% ============================================================================
\section{A Concrete Example}
% ============================================================================

% Introduce hadlink subtly---not as a showcase, but as an illustration.
%
% "In a recent single-sprint project, I constrained myself to a ten-day timeline
% and a sharply defined problem: a URL redirection service with formally verified
% core logic. The limitation forced me to define non-goals early (no analytics,
% no user accounts, no custom aliases), choose conservative tools (SQLite, Warp),
% and freeze scope quickly. The result was a small but production-ready system
% with formal proof obligations rather than an expanding prototype."
%
% Key details to weave in:
% - 10 days from first commit to v1.0.0 (Jan 22 -- Feb 1, 2026)
% - Explicit non-goals documented and enforced in contribution policy
% - Scope cuts made visible: no web UI, no click tracking, no custom aliases
% - Design by invariant: "All stored URLs are canonicalized" as a system property
%   that guided every architectural decision
% - Phased roadmap (v0.1.0 -> v0.5.0 -> v1.0.0) within the sprint
%
% Tone: understated. The project is the example, not the point.
%
% TODO: Write 2-3 paragraphs.

% ============================================================================
\section{Lessons and Advice}
% ============================================================================

% Addressed to students and early-career engineers.
%
% Practical suggestions:
% - Pick a problem you can define in one sentence
% - Write your non-goals before your goals
% - Set a hard deadline and do not move it
% - Ship something real, even if small
% - Treat the constraint as the feature, not the limitation
%
% Broader point: the habit of scoping, cutting, and shipping builds a skill set
% that complements coursework. Neither alone is sufficient. Together, they
% approximate the reality of professional engineering more closely than either
% can alone.
%
% TODO: Write 2-3 paragraphs.

% ============================================================================
\section{Conclusion}
% ============================================================================

% Reinforce:
% - Coursework and sprint projects are complementary, not competing
% - Sprint projects train judgment, ownership, and restraint
% - These are the skills that distinguish an engineer who can build from one
%   who can decide what to build
%
% TODO: Write 1-2 paragraphs.

\end{document}
